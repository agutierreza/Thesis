\begin{abstractslongEnglish} 
This thesis discusses dynamic decision making applications for a set of problems. Two main lines can be distinguished. The first deals with supply chain management problems for perishable products while the second studies the design of vessel fleets upon performing maintenance operations at offshore wind farms. The inventory models for perishable products studied in this thesis consider a single-item, single-stock location and production planning over a finite time horizon. The decision making problem of scheduling the maintenance operations at offshore wind farms is treated as a supply chain problem type: the installation requires to schedule maintenance operations and attend failures in turbines during the planning horizon. A fleet of vessels needs to be selected to support these operations.  For this set of problems, decisions are not only dynamic, but are also made under uncertainty.

The main objectives of this thesis are the following: (1) to study which order policies are the most appropriate for the designed perishable lot sizing problems. In which cases an order policy gives an optimal solution?; (2) to analise the effect of using parallel computing to improve the performance of the algorithms derived designing policies for perishable lot sizing problems; (3) to explore how effective heuristics can be for dynamic decision making in lot sizing problems for perishables; (4) to elaborate an MILP model for selecting a fleet of vessels to support the maintenance operations at offshore wind farms; and (5), to design a heuristic for scheduling maintenance operations at offshore wind farms considering failures in turbines and weather uncertainty.

Each one of these objectives have been discussed in a separate chapter of this thesis. In the second Chapter, a stochastic programming model is presented for a practical production planning problem of a perishable product over a finite time horizon. A static policy is studied for that model.  Such policy proved to be optimal assuming a static uncertainty strategy, which is considered for instances with a long lead time. The third Chapter addresses the use of parallel computing for the algorithms developed in the previous Chapter. Two implementations were developed for heterogeneous platforms: a multi-GPU version using CUDA and a multicore version using Pthreads and MPI. For the first implementation the Monte Carlo simulation (the most demanding task) is parallelised. The multicore version showed a good speedup, after dealing with an initially unbalanced workload among processors. The fourth Chapter discusses the effectiveness of heuristics for a similar lot sizing problem for perishables. The classical Silver heuristic is extended for perishable products and an analytical and a simulation-based variant of the approach are introduced. The results of the heuristics are compared with the optimal solutions given by a derived SDP model for the problem, showing that the heuristics feature costs that are, on average, 5\% above the optimal cost for the simulation approach and 6\% for the analytical approximation. In the fifth Chapter, a MILP model to select the optimal fleet of vessels to operate the maintenance of an offshore wind farm is derived. The model is presented as a bi-level problem, selecting the optimal fleet on the first level and optimising the schedule of operations, using the fleet, on the second. Since this model is deterministic, as others in literature aiming to solve long time horizon problems using small time periods, the sixth Chapter address the question of how the anticipation of stochastic events such as the failures in turbines or the weather conditions affect the decision of the optimal vessel fleet. This Chapter presents a heuristic decision rule that  illustrates this effect.
\end{abstractslongEnglish} 